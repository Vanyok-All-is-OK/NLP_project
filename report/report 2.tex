\documentclass[12pt]{article}
\usepackage[utf8]{inputenc}
\usepackage[english,russian]{babel}
\usepackage[T2A]{fontenc}
\usepackage{amsmath}
\usepackage{amssymb}
\usepackage{amsfonts}
\usepackage{xcolor}
\usepackage{enumitem}
\usepackage[top=2cm, bottom=2cm, left=2cm, right=2.5cm]{geometry}
\usepackage{lastpage}
\usepackage{fancyhdr}
\usepackage{mathrsfs}
\usepackage{multicol}
\usepackage[hidelinks]{hyperref}
\usepackage{tikz}
\usepackage{wasysym}
\usepackage{amsmath}
\usepackage[most]{tcolorbox}
\usepackage{parskip}
\graphicspath{{images/}}
\usepackage{graphicx}
\usepackage{hyperref}
\usepackage{float}


\usepackage{fontspec}
\setmainfont{Helvetica}
% \setmainfont{CMU Bright}
% \setmainfont{CMU Serif}

\hypersetup{
    colorlinks=true,
    linkcolor=cyan, % blue
    filecolor=magenta,      
    urlcolor=cyan,
    pdftitle={HW1 Dmitry Uspenskiy},
    pdfpagemode=FullScreen,
}

\newcommand{\imgh}[3]
{
\begin{figure}[H]
\center{\includegraphics[width=#1]{#2}}
\caption{#3}
\label{ris:#2}
\end{figure}
}

\newcommand{\condition}[1]
{
\begin{tcolorbox}[enhanced jigsaw,
    sharp corners,
    boxrule=0.5pt, 
    colback=white!30!white,   
    borderline={0.5pt}{-2pt}{black,solid} % 0.5pt linewith, -2pt outside, black solid linestyle
]
#1
\end{tcolorbox}
}

% \setcounter{section}{-1} %Нумерация с 0
\hyphenpenalty=10000

\pagestyle{fancyplain}
\headheight 35pt
\rhead{\textbf{Выполнили:} Успенский Д. А. \\ Беляев И. А. \\ Карбаев С. А. \\ \textbf{Группа:} 208}
\chead{\textbf{\large КТ 1} \\ [3ex] }
\lhead{ФКН ВШЭ \\ Автоматическая Обработка Текста \\ Осенний семестр 2023 \\ } 
\lfoot{}
\cfoot{}
\rfoot{\small\thepage}
\headsep 3em

\begin{document}
\tableofcontents
\newpage

\section{Краткое описание контрольной точки 1}

Давайте немного вспомним какая наша цель.

Мы участвуем в соревновании по оценке семантической текстовой связи между предложениями. В датасете есть два предложения и их оценка. Метрикой качества является коэффициент ранговой корреляции Спирмена.

Для бейслайн решения мы использовали TF-IDF вместе с RandomForestClassifier. Получили корреляцию 0.435

Наша основная модель это Bert. Мы вычисляли для каждого предложения его Bert-эмбеддинги, затем оценивали схожесть между предложениями как косинусное сходство этих эмбеддингов. Его выбрали из-за простоты реализации, а также потому, что он улучшает анализ слов относительно базового решения организаторов. 

Для улучшение результатов мы решили использовать DistilBert, т.к. обучающая выборка была маленькой и более тяжелые модели на основе Bert (AlBert, RoBerta) могли переобучиться

\newpage

\section{Новая модель}

Мы использовали модель DistilBert из-за маленькой выборки входных данных. Она является более облегченной версией Bert, которая, практически, не теряет своей эффективности, но ощутимо увеличивает скорость. Из-за этого риск переобучиться будет ниже по сравнению с базовой и другими моделями. 

Также мы попытались на практике посмотреть переобучатся ли тяжелые модели, такие как AlBert и RoBerta. Они показали результат хуже чем DistilBert, предположительно, как раз из-за переобучения. 

\newpage

\section{Результаты на соревновании}

\imgh{17cm}{leaderboard 2.png}{Leaderboard}

Выделение синим - наш текущий результат 0.797 (User spoker)

Наш прошлый результат - 0.671 (User ivan belyaev)

\newpage

\section{Итоги}

В конечном итоге у нас получилось достичь около 80$\%$ точности ответа. Как ни странно, облегчение модели помогло нам улучшить результат. 

Большинство участников, которые нас превзошли, получили результаты буквально на пару процентов выше. От лучшего результата мы отстаем не более чем на $5\%$



\end{document}
